\chapter{Introduction}

\lama is a programming language developed by JetBrains Research for education purposes. Its general characteristics are:

\begin{itemize}
\item procedural with first-class functions~--- functions can be passed as arguments, placed in data structures,
  returned and constructed at runtime via closures mechanism;
\item with lexical static scoping;
\item strict~--- all arguments of function application are evaluated before function body;
\item imperative~--- variables can be re-assigned, function calls can have side effects;
\item untyped~--- no static type checking is performed;
\item with S-expressions and pattern-matching;
\item with user-defined infix operators, including those defined in local scopes;
\item with automatic memory management (garbage collection).
\end{itemize}

The name \lama is an acronym for $\lambda\textsc{-Algol}$ since the language has borrowed the syntactic shape of
operators from \textsc{Algol-68}; \textsc{Haskell}~\cite{haskell} and \textsc{OCaml}~\cite{ocaml} can be
mentioned as other languages of inspiration.

%\section{General Characteristic of the Language}

\begin{itemize}
\item procedural with first-class functions~--- functions can be passed as arguments, placed in data structures,
  returned and constructed at runtime via closures mechanism;
\item with lexical static scoping;
\item strict~--- all arguments of function application are evaluated before function's body;
\item imperative~--- variables can be re-assigned, function calls can have side effects;
\item untyped~--- no static type checking is performed;
\item supports S-expressions and pattern-matching;
\item supports user-defined infix operators, including those defined in local scopes;
\item with automatic memory management (garbage collection).
\end{itemize}

%\section{Notation}

Pairs and tuples:

\[
\inbr{\bullet,\,\bullet,\,\dots}
\]

Lists of elements of kind $X$:

\[
X^*
\]

Deconstructing lists into sublists:

\[
h\circ t
\]

This applies also to lists of length 1. Empty list is denoted

\[
  \epsilon
\]


For a mapping $f : X\to Y$ we use the following definition:

\[
f [x\gets y] = \lambda\,z\,.\,
\left\{
\begin{array}{rcl}
  y    &,& x = z \\
  f\;x &,& x\neq z
\end{array}
\right.
\]

Empty mapping (undefined everywhere) is denoted $\Lambda$, the domain of a mapping $f$~--- $\dom{f}$, and we abbreviate

\[
  \Lambda[x_1\gets y_1][x_2\gets y_2]\dots[x_k\gets y_k]
\]

as

\[
  [x_1\gets y_1,\,x_2\gets y_2,\,\dots,\,x_k\gets y_k]
\]

%\input{01.03.values}
