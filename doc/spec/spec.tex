\documentclass{book}

\usepackage{amssymb, amsmath}
\usepackage{alltt}
\usepackage{pslatex}
\usepackage{epigraph}
\usepackage{verbatim}
\usepackage{latexsym}
\usepackage{array}
\usepackage{comment}
\usepackage{makeidx}
\usepackage{listings}
\usepackage{indentfirst}
\usepackage{verbatim}
\usepackage{color}
\usepackage{url}
\usepackage{xspace}
\usepackage{hyperref}
\usepackage{stmaryrd}
\usepackage{amsmath, amsthm, amssymb}
\usepackage{graphicx}
\usepackage{euscript}
\usepackage{mathtools}
\usepackage{mathrsfs}
\usepackage{multirow,bigdelim}
\usepackage{subcaption}
\usepackage{placeins}
\usepackage{xspace}
\usepackage{ostap}
\usepackage{bm}

\makeatletter

\makeatother

\definecolor{shadecolor}{gray}{1.00}
\definecolor{darkgray}{gray}{0.30}

\def\transarrow{\xrightarrow}
\newcommand{\setarrow}[1]{\def\transarrow{#1}}

\def\padding{\phantom{X}}
\newcommand{\setpadding}[1]{\def\padding{#1}}

\def\subarrow{}
\newcommand{\setsubarrow}[1]{\def\subarrow{#1}}

\newcommand{\trule}[2]{\frac{#1}{#2}}
\newcommand{\crule}[3]{\frac{#1}{#2},\;{#3}}
\newcommand{\withenv}[2]{{#1}\vdash{#2}}
\newcommand{\trans}[3]{{#1}\transarrow{\padding{\textstyle #2}\padding}\subarrow{#3}}
\newcommand{\ctrans}[4]{{#1}\transarrow{\padding#2\padding}\subarrow{#3},\;{#4}}
\newcommand{\llang}[1]{\mbox{\lstinline[mathescape]|#1|}}
\newcommand{\pair}[2]{\inbr{{#1}\mid{#2}}}
\newcommand{\inbr}[1]{\left<{#1}\right>}
\newcommand{\highlight}[1]{\color{red}{#1}}
\newcommand{\ruleno}[1]{\eqno[\scriptsize\textsc{#1}]}
\newcommand{\rulename}[1]{\textsc{#1}}
\newcommand{\inmath}[1]{\mbox{$#1$}}
\newcommand{\lfp}[1]{fix_{#1}}
\newcommand{\gfp}[1]{Fix_{#1}}
\newcommand{\vsep}{\vspace{-2mm}}
\newcommand{\supp}[1]{\scriptsize{#1}}
\newcommand{\sembr}[1]{\llbracket{#1}\rrbracket}
\newcommand{\cd}[1]{\texttt{#1}}
\newcommand{\free}[1]{\boxed{#1}}
\newcommand{\binds}{\;\mapsto\;}
\newcommand{\dbi}[1]{\mbox{\bf{#1}}}
\newcommand{\sv}[1]{\mbox{\textbf{#1}}}
\newcommand{\bnd}[2]{{#1}\mkern-9mu\binds\mkern-9mu{#2}}
\newtheorem{lemma}{Lemma}
\newtheorem{theorem}{Theorem}
\newcommand{\meta}[1]{{\mathcal{#1}}}
\renewcommand{\emptyset}{\varnothing}
\newcommand{\dom}[1]{\mathtt{dom}\;{#1}}
\newcommand{\primi}[2]{\mathbf{#1}\;{#2}}
\newcommand{\lama}{$\lambda\mbox{\textsc{Algol}}$\xspace}
%\newcommand{\sial}{S\textit{\lower -.5ex\hbox{I}\kern -.1667em\lower .5ex\hbox {A}}\kern -.125emL\@\xspace}
\definecolor{light-gray}{gray}{0.90}
\newcommand{\graybox}[1]{\colorbox{light-gray}{#1}}

\newcommand{\defterm}[1]{\textit{#1}}
\newcommand{\nonterm}[1]{\textit{#1}}
\newcommand{\term}[1]{\graybox{#1}}
\newcommand{\token}[1]{\textsc{#1}}
\newcommand{\alt}{\s\mid\s}
\newcommand{\s}{\:\:}

\lstdefinelanguage{lama}{
keywords={fun, case, esac, do, od, if, then, else, elif, fi, skip, repeat, until, for, local},
sensitive=true,
%basicstyle=\small,
commentstyle=\scriptsize\rmfamily,
keywordstyle=\ttfamily\bfseries,
identifierstyle=\ttfamily,
basewidth={0.5em,0.5em},
columns=fixed,
fontadjust=true,
literate={->}{{$\to$}}3,
morecomment=[s]{(*}{*)}
}

\lstset{
mathescape=true,
%basicstyle=\small,
identifierstyle=\ttfamily,
keywordstyle=\bfseries,
commentstyle=\scriptsize\rmfamily,
basewidth={0.5em,0.5em},
fontadjust=true,
escapechar=!,
language=lama
}

\sloppy

\title{\lama Language Definition}

\author{Dmitry Boulytchev}

\begin{document}

\maketitle

\chapter{Introduction}

\section{General Characteristic of the Language}

\begin{itemize}
\item procedural with first-class functions~--- functions can be passed as arguments, placed in data structures,
  returned and constructed at runtime via closures mechanism;
\item with lexical static scoping;
\item strict~--- all arguments of function application are evaluated before function's body;
\item imperative~--- variables can be re-assigned, function calls can have side effects;
\item untyped~--- no static type checking is performed;
\item supports S-expressions and pattern-matching;
\item supports user-defined infix operators, including those defined in local scopes;
\item with automatic memory management (garbage collection).
\end{itemize}

\section{Notation}

Pairs and tuples:

\[
\inbr{\bullet,\,\bullet,\,\dots}
\]

Lists of elements of kind $X$:

\[
X^*
\]

Deconstructing lists into sublists:

\[
h\circ t
\]

This applies also to lists of length 1. Empty list is denoted

\[
  \epsilon
\]


For a mapping $f : X\to Y$ we use the following definition:

\[
f [x\gets y] = \lambda\,z\,.\,
\left\{
\begin{array}{rcl}
  y    &,& x = z \\
  f\;x &,& x\neq z
\end{array}
\right.
\]

Empty mapping (undefined everywhere) is denoted $\Lambda$, the domain of a mapping $f$~--- $\dom{f}$, and we abbreviate

\[
  \Lambda[x_1\gets y_1][x_2\gets y_2]\dots[x_k\gets y_k]
\]

as

\[
  [x_1\gets y_1,\,x_2\gets y_2,\,\dots,\,x_k\gets y_k]
\]

\input{01.03.values}

\chapter{Abstract Syntax and Semantics}

\chapter{Concrete Syntax}

\begin{figure}[t]
  \[
    \begin{array}{rcl}
      \defterm{compilationUnit}  & : & \nonterm{import}^\star\s\nonterm{scopeExpression}\\
      \defterm{import}           & : & \term{import}\s\token{UIDENT}\s\term{;}
    \end{array}
  \]
  \caption{Compilation unit concrete syntax}
\end{figure}

\begin{figure}[t]
  \[
    \begin{array}{rcll}
      \defterm{expression}        & : & \nonterm{basicExpression}\s(\s\term{;}\s\nonterm{expression}\s)&\\
      \defterm{basicExpression}   & : & \nonterm{binaryExpression}&\\
      \defterm{binaryExpression}  & : & \nonterm{binaryOperand}\s\token{INFIX}\s\nonterm{binaryOperand}&\alt\\
                                  &   & \nonterm{binaryOperand}&\\
      \defterm{binaryOperand}     & : & \nonterm{binaryExpression}&\alt\\
                                  &   & [\s\term{-}\s]\s\nonterm{postfixExpression}&\\
      \defterm{postfixExpression} & : & \nonterm{primary}&\alt\\
                                  &   & \nonterm{postfixExpression}\s\term{(}\s[\s\nonterm{expression}\s(\s\term{,}\s\nonterm{expression}\s)^\star\s]\s\term{)}&\alt\\
                                  &   & \nonterm{postfixExpression}\s\term{[}\s\nonterm{expression}\s\term{]}&\alt\\
                                  &   & \nonterm{postfixExpression}\s\term{.}\s\term{length}&\alt\\
                                  &   & \nonterm{postfixExpression}\s\term{.}\s\term{string}&\\      

      \defterm{primary}           & : & \token{DECIMAL}&\alt\\
                                  &   & \token{STRING}&\alt\\
                                  &   & \token{CHAR}&\alt\\
                                  &   & \token{LIDENT}&\alt\\
                                  &   & \term{true}&\alt\\
                                  &   & \term{false}&\alt\\
                                  &   & \term{infix}\s\token{INFIX}&\alt\\
                                  &   & \term{skip}&\alt\\
                                  &   & \term{fun}\s\term{(}\s\nonterm{functionArguments}\s\term{)}\s\nonterm{functionBody}&\alt\\
                                  &   & \term{\{}\s\nonterm{scopeExpression}\s\term{\}}&\alt\\
                                  &   & \nonterm{listExpression}&\alt\\
                                  &   & \nonterm{arrayExpression}&\alt\\
                                  &   & \nonterm{S-expression}&\alt\\
                                  &   & \nonterm{ifExpression}&\alt\\
                                  &   & \nonterm{whileExpression}&\alt\\
                                  &   & \nonterm{repeatExpression}&\alt\\
                                  &   & \nonterm{forExpression}&\alt\\
                                  &   & \nonterm{caseExpression}&\alt\\
                                  &   & \term{(}\s\nonterm{expression}\s\term{)}&
    \end{array}
  \]
  \caption{Expression concrete syntax}
\end{figure}


\begin{figure}[t]
  \[
    \begin{array}{rcl}
      \defterm{scopeExpression}                & : & \nonterm{definition}^\star\s\nonterm{expression}\\
      \defterm{definition}                     & : & \nonterm{variableDefinition}\alt\nonterm{functionDefinition}\alt\nonterm{infixDefinition}\\
      \defterm{variableDefinition}             & : & (\s\term{local}\alt\term{public}\s)\s\nonterm{variableDefinitionSequence}\s\term{;}\\
      \defterm{variableDefinitionSequence}     & : & \nonterm{variableDefinitionSequenceItem}\s(\s\term{,}\s\nonterm{variableDefinitionSequenceItem}\s)^\star\\
      \defterm{variableDefinitionSequenceItem} & : & \token{LIDENT}\s[\s\term{=}\s\nonterm{basicExpression}\s]\\
      \defterm{functionDefinition}             & : & [\s\term{public}\s]\s\term{fun}\s\token{LIDENT}\s\term{(}\s\nonterm{functionArguments}\s\term{)}\s\nonterm{functionBody}\\
      \defterm{functionArguments}              & : & [\s\token{LIDENT}\s(\s\term{,}\s\token{LIDENT}\s)^\star\s]\\
      \defterm{functionBody}                   & : & \term{\{}\s\nonterm{scopeExpression}\s\term{\}}
    \end{array}
  \]
  \caption{Scope expression concrete syntax}
\end{figure}


\begin{figure}[t]
  \[
    \begin{array}{rcll}
      \defterm{pattern}         & : & \nonterm{consPattern}\alt\nonterm{simplePattern}&\\
      \defterm{consPattern}     & : & \nonterm{simplePattern}\s\term{:}\s\nonterm{pattern}&\\
      \defterm{simplePattern}   & : & \nonterm{wildcardPattern} & \alt\\
                                &   & \nonterm{S-exprPattern} & \alt \\
                                &   & \nonterm{arrayPattern} & \alt \\
                                &   & \nonterm{listPattern} & \alt \\
                                &   & \token{LIDENT}\s[\s\term{@}\s\nonterm{pattern} \s] & \alt \\
                                &   & [\s\term{-}\s]\s\token{DECIMAL}& \alt \\
                                &   & \token{STRING} & \alt \\
                                &   & \token{CHAR} & \alt \\
                                &   & \term{true} & \alt \\
                                &   & \term{false} & \alt \\
                                &   & \term{\#}\s\term{boxed} & \alt \\
                                &   & \term{\#}\s\term{unboxed} & \alt \\
                                &   & \term{\#}\s\term{string} & \alt \\
                                &   & \term{\#}\s\term{array} & \alt \\
                                &   & \term{\#}\s\term{sexp} & \alt \\
                                &   & \term{\#}\s\term{fun} & \alt \\
                                &   & \term{(}\s\nonterm{pattern}\s\term{)} & \\
      \defterm{wildcardPattern} & : & \term{\_} &\\
      \defterm{S-exprPattern}   & : & \token{UIDENT}\s[\s\term{(}\s\nonterm{pattern}\s(\s\term{,}\s\nonterm{pattern})^\star\s\term{)}\s] &\\
      \defterm{arrayPattern}    & : & \term{[}\s[\s\nonterm{pattern}\s(\s\term{,}\s\nonterm{pattern})^\star\s]\s\term{]} &\\
      \defterm{listPattern}     & : & \term{\{}\s[\s\nonterm{pattern}\s(\s\term{,}\s\nonterm{pattern})^\star\s]\s\term{\}} &
    \end{array}
  \]
  \caption{Pattern concrete syntax}
\end{figure}

\begin{figure}[t]
  \[
    \begin{array}{rcll}
      \defterm{ifExpression}  & : & \term{if}\s\nonterm{expression}\s\term{then}\s\nonterm{scopeExpression}\s[\s\nonterm{elsePart}\s]\s\term{fi}&\\
      \defterm{elsePart}      & : & \term{elif}\s\nonterm{expression}\s\term{then}\s\nonterm{scopeExpression}\s[\s\nonterm{elsePart}\s]&\alt\\
                              &   & \term{else}\s\nonterm{scopeExpression}&
    \end{array}
  \]
  \caption{If-expression concrete syntax}
\end{figure}

\begin{figure}[t]
  \[
    \begin{array}{rcl}
      \defterm{whileExpression}  & : & \term{while}\s\nonterm{expression}\s\term{do}\s\nonterm{scopeExpression}\s\term{od}\\
      \defterm{repeatExpression} & : & \term{repeat}\s\nonterm{scopeExpression}\s\term{until}\s\nonterm{basicExpression}\\
      \defterm{forExpression}    & : & \term{for}\s\nonterm{expression}\s\term{,}\s\nonterm{expression}\s\term{,}\s\nonterm{expression}\\
                                 &   & \term{do}\nonterm{scopeExpresssion}\s\term{od}
    \end{array}
  \]
  \caption{Loop expressions concrete syntax}  
\end{figure}

\begin{figure}[t]
  \[
    \begin{array}{rcl}
      \defterm{arrayExpression} & : & \term{[}\s[\s\nonterm{expression}\s(\s\term{,}\s\nonterm{expression}\s)^\star\s]\s\term{]}\\
      \defterm{listExpression}  & : & \term{\{}\s[\s\nonterm{expression}\s(\s\term{,}\s\nonterm{expression}\s)^\star\s]\s\term{\}}\\
      \defterm{S-expression}    & : & \token{UIDENT}\s[\s\term{(}\s\nonterm{expression}\s[\s(\s\term{,}\s\nonterm{expression}\s)^\star\s]\term{)}\s]
    \end{array}
  \]
  \caption{Array, list, and S-expressions concrete syntax}  
\end{figure}


\begin{figure}[t]
  \[
    \begin{array}{rcl}
      \defterm{caseExpression}  & : & \term{case}\s\nonterm{expression}\s\term{of}\s\nonterm{caseBranches}\s\term{esac}\\
      \defterm{caseBranches}    & : & \nonterm{caseBranch}\s[\s(\s\term{$\mid$}\s\nonterm{caseBranch}\s)^\star\s]\\
      \defterm{caseBranch}      & : & \nonterm{pattern}\s\term{$\rightarrow$}\s\nonterm{scopeExpression}
    \end{array}
  \]
  \caption{Case-expression concrete syntax}
\end{figure}




\end{document}
